\documentclass[a4paper,twoside,10pt]{article}

\usepackage[utf8]{inputenc}
\usepackage{geometry}
\usepackage{listings}
\usepackage{lstlangarm}
\usepackage{multirow}
\usepackage{multicol}
\usepackage{graphicx}
\usepackage{xcolor}


\lstset{
	language=[ARM]Assembler,
	tabsize=2,
	backgroundcolor = \color{gray!10},
}

\geometry{
	top 	= .75cm,
	bottom 	= .5cm,
	inner 	= .75cm,
	outer 	= 1cm,
}


\usepackage{tgcursor}
%\usepackage{showframe}


\usepackage[T1]{fontenc}
\begin{document}
\small

	\def\instruction#1#2#3{&\texttt{#1} & #2 &\small #3 \\}
	
	\def\cdS{\cd\option{\texttt{S}}}
	\def\cdB{\cd\option{\texttt{B}}}
	\def\cdum{\rmfamily\itshape\cd um}
	\def\cd{\option{\rmfamily\emph{cd}}}
	
	\def\svarkword#1{\emph{\textrm{#1}}}
	\def\kword#1{\texttt{#1}}

	\def\option#1{\textcolor{gray}{#1}}	

	\newcommand{\reg}[1][]{\svarkword{reg\textsubscript{#1}}}
	\newcommand{\imm}[1][]{\svarkword{imm}\textsubscript{\textrm{\upshape{#1}}}}
	\def\shift{\svarkword{shift}}
	\def\arg{\svarkword{arg}}
	\def\mem{\svarkword{mem}}
	\def\mreg{\svarkword{mreg}}
	\def\ss#1{\svarkword{\textsubscript{#1}}}
	
	
	\renewcommand{\arraystretch}{1}


	\begin{table}[p!]\centering
				
		\begin{tabular}{p{.5cm}l@{\hspace{2mm}}l@{\hspace{.5cm}}l}
				%
			\multicolumn{4}{c}{\bfseries ARM Instructions} \\
			\hline
				%
			\parbox[t]{2mm}{\multirow{12}{2.5cm}{\rotatebox[origin=c]{90}{\parbox{5cm}{\bfseries\centering Arithmetic}}}}%
				\instruction{ADD\cdS\textsuperscript{$\dagger$}}{\reg, \reg, \arg}{add}
				\instruction{SUB\cdS}{\reg, \reg, \arg}{subtract}
				\instruction{RSB\cdS}{\reg, \reg, \arg}{subtract reversed operands}
				\instruction{ADC\cdS}{\reg, \reg, \arg}{add both operands and carry flag}
				\instruction{SBC\cdS}{\reg, \reg, \arg}{subtract both operands and adds carry flag${}-1$}
				\instruction{RSC\cdS}{\reg, \reg, \arg}{reverse subtract both operands and adds carry flag${}-1$}
				\instruction{MUL\cdS}{\reg\ss{d}, \reg\ss{m}, \reg[s]}{multiply \reg[m] and \reg[s], places lower 32 bits into \reg[d]}
				\instruction{MLA\cdS}{\reg[d], \reg[m], \reg[s], \reg[n]}{places lower 32 bits of \reg[m] · \reg[s] + \reg[n] into \reg[d]}
				\instruction{UMULL\cdS}{\reg[lo], \reg[hi], \reg[m], \reg[s]}{multiply \reg[m] and \reg[s] place 64-bit unsigned result into \{\reg[hi], \reg[lo]\}}
				\instruction{UMLAL\cdS}{\reg[lo], \reg[hi], \reg[m], \reg[s]}{place unsigned \reg[m] ${}\cdot{}$ \reg[s] + \{\reg[hi], \reg[lo]\} into \{\reg[hi], \reg[lo]\}}
				\instruction{SMULL\cdS}{\reg[lo], \reg[hi], \reg[m], \reg[s]}{multiply \reg[m] and \reg[s], place 64-bit signed result into \{\reg[hi], \reg[lo]\}}
				\instruction{SMLAL\cdS}{\reg[lo], \reg[hi], \reg[m], \reg[s]}{place signed \reg[m] ${}\cdot{}$ \reg[s] ${}+{}$ \{\reg[hi], \reg[lo]\} into \{\reg[hi], \reg[lo]\}}
				
			\hline
			\parbox[t]{2mm}{\multirow{4}{2.5cm}{\rotatebox[origin=c]{90}{\parbox{1.25cm}{\bfseries\centering Bitwise logic}}}}%
				\instruction{AND\cdS}{\reg, \reg, \arg}{bitwise AND}
				\instruction{ORR\cdS}{\reg, \reg, \arg}{bitwise OR}
				\instruction{EOR\cdS}{\reg, \reg, \arg}{bitwise exclusive-OR}
				\instruction{BIC\cdS}{\reg, \reg[a], \arg\ss{b}}{bitwise \reg[a] AND (NOT \arg\ss{b})}
			\hline
			\parbox[t]{2mm}{\multirow{4}{2.5cm}{\rotatebox[origin=c]{90}{\parbox{1.25cm}{\bfseries\centering Comp-\linebreak arison}}}}%
				\instruction{CMP\cd}{\reg, \arg}{update flags based on subtraction}
				\instruction{CMN\cd}{\reg, \arg}{update flags based on addition}
				\instruction{TST\cd}{\reg, \arg}{update flags based on bitwise AND}
				\instruction{TEQ\cd}{\reg, \arg}{update flags based on bitwise exclusive-OR}
			\hline
			\parbox[t]{2mm}{\multirow{2}{1cm}{\hspace{-1.7cm}\bfseries Data movement}}%
				\instruction{MOV\cdS}{\reg, \arg}{copy argument}
				\instruction{MVN\cdS}{\reg, \arg}{copy bitwise NOT of argument}
			\hline
			\parbox[t]{2mm}{\multirow{5}{2.5cm}{\rotatebox[origin=c]{90}{\parbox{1.75cm}{\bfseries\centering Memory access}}}}%
				\instruction{LDR\cdB\textsuperscript{$\ddagger$}}{\reg, \mem}{loads word/ byte/ half from memory into a register}
				\instruction{STR\cdB}{\reg, \mem}{stores word/ byte/ half to memory from a register}
				\instruction{LDM\cdum}{\reg\option{!}, \mreg}{loads into multiple registers}
				\instruction{STM\cdum}{\reg\option{!}, \mreg}{stores multiple registers}
				\instruction{SWP\cdB}{\reg[d], \reg[m], [\reg[n]]}{copies \reg[m] to memory at \reg[n], old value at address \reg[n] to \reg[d]}
			\hline
			\parbox[t]{2mm}{\multirow{5}{2.5cm}{\rotatebox[origin=c]{90}{\parbox{1cm}{\bfseries\centering Branch-ing}}}}%
				\instruction{B\cd}{\imm[24]}{branch to \imm[24] words away}
				\instruction{BL\cd}{\imm[24]}{copy PC to LR, then branch}
				\instruction{BX\cd}{\reg}{copy \reg\ to PC, and exchange instruction sets (T flag := \reg{[}0])}
				\instruction{SWI\cd}{\imm[24]}{software interrupt}
			\hline
			& \multicolumn{3}{r}{\footnotesize \textsuperscript{$\dagger$} \kword{S} = set condition flags\hspace{1cm}  \textsuperscript{$\ddagger$} \kword{B} = byte, can be replaced by \kword{H} for half word(2 bytes)} \\
		\end{tabular}
			
	\end{table}
	
	\def\code#1#2#3{\texttt{#1} & \small #2 & #3\\}
	
	
	\begin{table}[p!]\centering
				
		
		\begin{tabular}{l@{\hspace{5mm}}ll}
				%
			\multicolumn{3}{c}{\bfseries \emph{cd}: condition code} \\
			\hline
				%
				\code{AL \textrm{or omitted}}{always}{(ignored)}
				\code{EQ}{equal}{Z = 1}
				\code{NE}{not equal}{Z = 0}
				\code{CS}{carry set (same as \texttt{HS})}{C = 1}
				\code{CC}{carry clear (same as \texttt{LO})}{C = 0}
				\code{MI}{minus}{N = 1}
				\code{PL}{positive or zero}{N = 0}
				\code{VS}{overflow}{V = 1}
				\code{VC}{no overflow}{V = 0}
				\code{HS}{unsigned higher or same}{C = 1}
				\code{LO}{unsigned lower}{C = 0}
				\code{HI}{unsigned higher}{C = 1 $\wedge$ Z = 0}
				\code{LS}{unsigned lower or same}{C = 0 $\vee$ Z = 1}
				\code{GE}{signed greater than or equal}{N = V}
				\code{LT}{signed less than}{N $\neq$ V}
				\code{GT}{signed greater than}{Z = 0 $\wedge$ N = V}
				\code{LE}{signed less than or equal}{Z = 1 $\vee$ N $\neq$ V}\hline
				
		\end{tabular}
		\hfill
		\def\code#1#2{\texttt{#1} &\small #2 \\}
		\begin{tabular}{l@{\hspace{5mm}}l}
			%
			\multicolumn{2}{c}{\bfseries \emph{um}: update mode} \\
			\hline
					%
			\code{FA / IB}{ascending, starting from \reg}
			\code{EA / IA}{ascending, starting from \reg ${}+{} 4$}
			\code{FD / DB}{descending, starting from \reg}
			\code{ED / DA}{descending, starting from \reg ${}- 4$}\hline
			&\\
			&\\
			\multicolumn{2}{c}{\bfseries \emph{reg}: register} \\
			\hline
					%
			\code{R0 \textrm{to} R15}{register according to number}
			\code{SP}{register 13}
			\code{LR}{register 14}
			\code{PC}{register 15}\hline
			&\\
			&\\
			\multicolumn{2}{c}{\bfseries \emph{arg}: right-hand argument} \\
			\hline
				%
			\code{\#\imm[8]}{immediate on 8 bits, possibly rotated right}
			\code{\reg}{register}
			\code{\reg, \shift}{register shifted by distance}\hline
		\end{tabular}
				
	\end{table}
	
	\def\code#1#2{\texttt{#1} & \small #2 \\}
	\begin{table}
	
		\begin{tabular}{l@{\hspace{3mm}}l}
					%
			\multicolumn{2}{c}{\bfseries \emph{shift}: shift register value} \\
			\hline
			%
			\code{LSL \#\imm[5]}{shift left 0 to 31}
			\code{LSR \#\imm[5]}{logical shift right 1 to 32}
			\code{ASR \#\imm[5]}{arithmetic shift right 1 to 32}
			\code{ROR \#\imm[5]}{rotate right 1 to 31}
			\code{RRX}{rotate carry bit into top bit}
			\code{LSL \reg}{shift left by register}
			\code{LSR \reg}{logical shift right by register}
			\code{ASR \reg}{arithmetic shift right by register}
			\code{ROR \reg}{rotate right by register}\hline
			\multicolumn{2}{r}{}\\
		\end{tabular}
		\hfill
		\begin{tabular}{l@{\hspace{3mm}}l}
					%
			\multicolumn{2}{c}{\bfseries \emph{mem}: memory address} \\
			\hline
			%
			\code{[\reg,\#\textcolor{gray}{$\pm$}\imm[12]]}{\reg\ offset by constant}
			\code{[\reg,\textcolor{gray}{$\pm$}\reg]}{\reg\ offset by variable bytes}
			\code{[\reg[a],\textcolor{gray}{$\pm$}\reg[b],\shift]}{\reg[a]\ offset by shifted variable \reg[b]\ \textsuperscript{$\dagger$}}
			\code{[\reg,\#\textcolor{gray}{$\pm$}\imm[12]]!}{update \reg\ by constant, then access memory}
			\code{[\reg,\textcolor{gray}{$\pm$}\reg]!}{update \reg\ by variable bytes, then access memory}
			\code{[\reg,\textcolor{gray}{$\pm$}\reg,\shift]!}{update \reg\ by shifted variable, then access memory\ \textsuperscript{$\dagger$}}
			\code{[\reg],\#\textcolor{gray}{$\pm$}\imm[12]}{access address \reg, then update \reg\ by offset}
			\code{[\reg],\textcolor{gray}{$\pm$}\reg}{access address \reg, then update \reg\ by variable}
			\code{[\reg],\textcolor{gray}{$\pm$}\reg,\shift}{access address \reg, then update \reg\ by shifted variable\ \textsuperscript{$\dagger$}}\hline
			\multicolumn{2}{r}{\footnotesize \textsuperscript{$\dagger$} shift distance must be by constant} \\
		\end{tabular}
	
	\end{table}
	



\end{document}


